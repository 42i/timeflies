\documentclass[11pt]{article}
\usepackage{xunicode}
\usepackage{fontspec}
\usepackage{xltxtra}
\usepackage{float}
\usepackage{fancyvrb}
\defaultfontfeatures{Scale=MatchLowercase}
\setmainfont[Mapping=tex-text,Ligatures=TeX,Numbers=OldStyle]{Linux Libertine O}
\setsansfont[Mapping=tex−text,Numbers=OldStyle] {Linux Biolinum O}
\setmonofont[Mapping=tex-text,Scale=0.8] {DejaVu Sans Mono}

\newcommand{\timeflies}{\emph{TimeFlies}}

%\setromanfont{Linux Libertine O}

\begin{document}

\DefineVerbatimEnvironment{inputfile}{Verbatim}
   {%frame=single,
    %framesep=1mm,
    numbers=left,
    baselinestretch=0.8,
    labelposition=topline}

%\newfloat{example}{thp}{lof}[section]
%\floatname{example}{Example}

\title{\timeflies\ -- A Tool for Time Keeping}
\author{J\"org Bullmann}
\maketitle
\tableofcontents
\newpage

\section{What is \timeflies\ About?}

Do you need to account for the time you spend at work? What project or work package have you been working on? Do you want to keep track of the hours you work? How much leave have you got left for this year?

Do you keep a daily work log containing things you did, problems you solved, some kind of to do list?

Do you want to make an estimate of effort for a project or work package? Would you like to break down those things into smaller items and possibly break down those again too?

\timeflies\ can help you with this.

\section{Tutorial by Example}

In this section we will look at a number of use cases. All \timeflies\ data is kept in simple text files. So all your data is always easily accessible to you and the format itself is quite human-readable. Moreover it can easily be version controlled.

\subsection{Recording Time}

To record your work time keep a work log file with \verb-day- lines specifying the dates and in and out times telling when you arrived at work and when you left. The times can be given in decimals or in hours and minutes:

\begin{inputfile}
day 2012-09-19 8:30 17:15
day 2012-09-18 8.75 17.75
day 2012-09-17 8 18
\end{inputfile}

Do you need to account for breaks you are taking? Use \verb-off- instructions to state periods of time in a day of work during which you were not actually working:

\begin{inputfile}
day 2012-09-19 8.5 17.25, off 0.5
day 2012-09-18 8.75 17.75, off 0.75
day 2012-09-17 8 18, off 0.5, off 0.25
\end{inputfile}

This last file is equivalent to the following:

\begin{inputfile}
day 2012-09-19 8.5 17.25
off 0.5
day 2012-09-18 8.75 17.75
off 0.75
day 2012-09-17 8 18
off 0.5
off 0.25
\end{inputfile}

This last example illustrates the notion of a \verb-day-\emph{-block}: a \verb-day-\emph{-block} extends
from one \verb-day--keyword to the next and everything inside this day block is part of that day.

The days in the file do not need to be listed chronologically. You could e.g.\ list the days in reverse order so that the present is always at the top of the file.

If you want to mask out part of your log temporarily you can use the \verb-#- source comment marker. \timeflies\ ignores the \verb-#- and everything following it until the end of line. It works just the same as e.g.\ a Python comment.

\begin{inputfile}
day 2012-09-19 8.5 17.25
off 0.5

# day 2012-09-18 8.75 17.75
# off 0.75

day 2012-09-17 8 18
off 0.5
off 0.25
\end{inputfile}

\subsection{Keeping Notes in the Log}

Do you want to keep notes about your work in the same place as you keep the time information? Use \emph{log comment} lines like in this file:

\begin{inputfile}
day 2012-09-19 8.5 17.25
-- updated regression tests
off 0.5
-- fixed build scripts

day 2012-09-18 8.75 17.75
-- wrote unit test to reproduce problem report 2012-0098
-- fixed problem report 2012-0098
off 0.75
-- added HTML output option to object dumper
-- discussed implications of Java 1.7 rollout

day 2012-09-17 8 18
-- weekly team meeting
off 0.5
-- monthly quality task force 
off 0.25
-- code review: server side includes
\end{inputfile}

A log comment line starts in double dash characters \verb=--= and one or more space characters. All text following these characters until the end of the line (or until a \verb-#- source commend marker) with trailing spaces removed constitute the recorded log comment.

Now what can you do with such a file? Assume the above work log file's name is \verb:work-log.fly:, then option \verb:-t: tells \timeflies\ to calculate your work times.

\begin{inputfile}
> timeflies.py -t work-log.fly 
Time at work overview (all):
2012-09-17, Mon:  9.25 worked, --.-- leave, --.-- ill
2012-09-18, Tue:  8.25 worked, --.-- leave, --.-- ill
2012-09-19, Wed:  8.25 worked, --.-- leave, --.-- ill
   week 2012-38: 25.75 worked, --.-- leave, --.-- ill
  month 2012-09: 25.75 worked, --.-- leave, --.-- ill
          total: 25.75 worked, --.-- leave, --.-- ill
\end{inputfile}

To include the log comments in this output, use the \verb=-C= option:

\begin{inputfile}
> timeflies.py -t -C work-log.fly 
Time at work overview (all):
2012-09-17, Mon:  9.25 worked, --.-- leave, --.-- ill
             -- weekly team meeting
             -- monthly quality task force
             -- code review: server side includes
2012-09-18, Tue:  8.25 worked, --.-- leave, --.-- ill
             -- wrote unit test to reproduce problem report 2012-0098
             -- fixed problem report 2012-0098
             -- added HTML output option to object dumper
             -- discussed implications of Java 1.7 rollout
2012-09-19, Wed:  8.25 worked, --.-- leave, --.-- ill
             -- updated regression tests
             -- fixed build scripts
   week 2012-38: 25.75 worked, --.-- leave, --.-- ill
  month 2012-09: 25.75 worked, --.-- leave, --.-- ill
          total: 25.75 worked, --.-- leave, --.-- ill
\end{inputfile}

\subsection{Logging Activities}

Log comments are a good way to keep track of things you don't want to forget and have accessible and also aligned with your work time line. Log comments have no work effort assigned to them, though. So you cannot use them in any way for calculations of effort spent.

To create connections between your working time and work packages y

You use \emph{work packages} and \emph{activities} to connect the time you work to the work packages you work on: first, you define your work packages, then you use \emph{activity} lines in the day blocks instead of log comment lines.

A work package definition is a line starting with the keyword \verb:work-package: (or the short version \verb:wp:) followed by a work package name.

An activity line starts with a single dash character \verb:-: followed by one or more spaces. This is followed by a work package id and a duration. This is optionally followed by a comma and some activity comment.

See below the converted example work log file.

\begin{inputfile}
work-package regression-tests
work-package meetings
work-package quality-task-force
work-package problem-reports
work-package development
work-package other

day 2012-09-19 8.5 17.25
- regression-tests 4, updated
off 0.5
- other 3.5, fixed build scripts

day 2012-09-18 8.75 17.75
- problem-reports 2, wrote unit test to reproduce problem report 2012-0098
- problem-reports 2.5, fixed problem report 2012-0098
off 0.75
- development 3, added HTML output option to object dumper
- other 1, discussed implications of Java 1.7 rollout

day 2012-09-17 8 18
- meetings 2.0, weekly team meeting
off 0.5
- quality-task-force 6 
off 0.25
- other 1.25, code review: server side includes
\end{inputfile}

Option \verb:-w: tells \timeflies\ to calculate the times you have been working on the different work packages:

\begin{inputfile}
> timeflies.py -w work-log.fly 
Work package summary (all):
 25.25 : ALL
      4.00 : regression-tests
      2.00 : meetings
      6.00 : quality-task-force
      4.50 : problem-reports
      3.00 : development
      5.75 : other
\end{inputfile}

To also show the activities contributing to the different work packages, use option \verb:-a::

\begin{inputfile}
> timeflies.py -w -a work-log.fly 
Work package summary (all):
 25.25 : ALL
      4.00 : regression-tests
             - 2012-09-19 4.0, updated
      2.00 : meetings
             - 2012-09-17 2.0, weekly team meeting
      6.00 : quality-task-force
             - 2012-09-17 6.0
      4.50 : problem-reports
             - 2012-09-18 2.0, wrote unit test to reproduce problem report 2012-0098
             - 2012-09-18 2.5, fixed problem report 2012-0098
      3.00 : development
             - 2012-09-18 3.0, added HTML output option to object dumper
      5.75 : other
             - 2012-09-19 3.5, fixed build scripts
             - 2012-09-18 1.0, discussed implications of Java 1.7 rollout
             - 2012-09-17 1.25, code review: server side includes
\end{inputfile}

To check whether you have allocated all your working time to work packages, use option \verb:-c:

\begin{inputfile}
> timeflies.py -c work-log.fly 
Day check (all):
*** 2012-09-18, Tue : worked =  8.25, tasked =  8.50, delta =  0.25
*** 2012-09-19, Wed : worked =  8.25, tasked =  7.50, delta = -0.75
\end{inputfile}

This shows that on two days the time at work and the time worked on work packages are differing.

\subsection{Work Package Breakdown}

In the previous section, work packages have been defined as simple, atomic, named items. A work package can be subdivided and refined in hierarchical way. See the following example.

\begin{inputfile}
wp mighty-digester, digest inputs of all sorts and spit them back out
    input
        xml
        json
        dotted-text
        binary
    processing
        statistics
        phase-1, rough break-down
        phase-2, particle recombination
        phase-3, regrouping and amalgamation
    output
        xml
        json
        text
        binary
    mmi
        gui
        cmdline
\end{inputfile}

The items in this work package hierarchy can be referred to in activity lines as dot separated work package path names. In above work package definition examples for such paths are \verb:mighty-digester.input.xml:, \verb:mighty-digester.processing.phase-1:.

Following, a piece of work log for the above project.

\begin{inputfile}
day 2012-07-01 8 17, off 1
- mighty-digester.input.xml 4
- mighty-digester.output.xml 3
- mighty-digester.mmi.cmdline 1
day 2012-07-02 8 17, off 1
- mighty-digester.input.json 5
- mighty-digester.processing.statistics 1.5
- mighty-digester.mmi.cmdline 1.5
day 2012-07-03 8 17, off 1
- mighty-digester.input.xml 4
- mighty-digester.output.xml 3
- mighty-digester.output.text 0.5
- mighty-digester.mmi.cmdline 0.5
\end{inputfile}

Assuming the work package defintions are kept in file \verb:prj-mighty-digester.fly: and the work log itself in \verb:work-log.fly:, the work package summary can be calculated as follows:

\begin{inputfile}
> timeflies.py -w prj-mighty-digester.fly work-log.fly 
Work package summary (all):
 24.00 : ALL
     24.00 : mighty-digester -- digest inputs of all sorts and spit them back out
         13.00 : input
              8.00 : xml
              5.00 : json
          1.50 : processing
              1.50 : statistics
          6.50 : output
              6.00 : xml
              0.50 : text
          3.00 : mmi
              3.00 : cmdline
\end{inputfile}

\subsection{Monitor Project Progress}

\section{Reference}


\end{document}

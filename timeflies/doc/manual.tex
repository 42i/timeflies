\documentclass[11pt]{article}
\usepackage{xunicode}
\usepackage{fontspec}
\usepackage{xltxtra}
\usepackage{float}
\usepackage{fancyvrb}
\defaultfontfeatures{Scale=MatchLowercase}
\setmainfont[Mapping=tex-text,Ligatures=TeX,Numbers=OldStyle]{Linux Libertine O}
\setsansfont[Mapping=tex−text,Numbers=OldStyle] {Linux Biolinum O}
\setmonofont[Mapping=tex-text,Scale=0.8] {DejaVu Sans Mono}

\newcommand{\timeflies}{\emph{TimeFlies}}

%\setromanfont{Linux Libertine O}

\begin{document}

\DefineVerbatimEnvironment{inputfile}{Verbatim}
   {%frame=single,
    %framesep=1mm,
    numbers=left,
    baselinestretch=0.8,
    labelposition=topline}

%\newfloat{example}{thp}{lof}[section]
%\floatname{example}{Example}

\title{\timeflies\ -- A Tool for Time Keeping}
\author{J\"org Bullmann}
\maketitle
\tableofcontents
\newpage

\section{What is \timeflies\ About?}

Do you need to account for the time you spend at work? What project or work package have you been working on? Do you want to keep track of the hours you work? How much leave have you got left for this year?

Do you keep a daily work log containing things you did, problems you solved, some kind of to do list?

Do you want to make an estimate of effort for a project or work package? Would you like to break down those things into smaller items and possibly break down those again too?

\timeflies\ can help you with this.

\section{Tutorial by Example}

In this section we will look at a number of use cases. All \timeflies\ data is kept in simple text files. So all your data is always easily accessible to you and the format itself is quite human-readable. Moreover it can easily be version controlled.

\subsection{Recording Time}

To record your work time keep a work log file with \verb-day- lines specifying the dates and in and out times telling when you arrived at work and when you left. The times can be given in decimals or in hours and minutes:

\begin{inputfile}
day 2012-09-19 8:30 17:15
day 2012-09-18 8.75 17.75
day 2012-09-17 8 18
\end{inputfile}

Do you need to account for breaks you are taking? Use \verb-off- instructions to state periods of time in a day of work during which you were not actually working:

\begin{inputfile}
day 2012-09-19 8.5 17.25, off 0.5
day 2012-09-18 8.75 17.75, off 0.75
day 2012-09-17 8 18, off 0.5, off 0.25
\end{inputfile}

This last file is equivalent to the following:

\begin{inputfile}
day 2012-09-19 8.5 17.25
off 0.5
day 2012-09-18 8.75 17.75
off 0.75
day 2012-09-17 8 18
off 0.5
off 0.25
\end{inputfile}

This last example illustrates the notion of a \verb-day-\emph{-block}: a \verb-day-\emph{-block} extends
from one \verb-day--keyword to the next and everything inside this day block is part of that day.

The days in the file do not need to be listed chronologically. You could e.g.\ list the days in reverse order so that the present is always at the top of the file.

If you want to mask out part of your log temporarily you can use the \verb-#- source comment marker. \timeflies\ ignores the \verb-#- and everything following it until the end of line. It works just the same as e.g.\ a Python comment.

\begin{inputfile}
day 2012-09-19 8.5 17.25
off 0.5

# day 2012-09-18 8.75 17.75
# off 0.75

day 2012-09-17 8 18
off 0.5
off 0.25
\end{inputfile}

\subsection{Keeping Notes in the Log}

Do you want to keep notes about your work in the same place as you keep the time information? Use \emph{log comment} lines like in this file:

\begin{inputfile}
day 2012-09-19 8.5 17.25
-- updated regression tests
off 0.5
-- fixed build scripts

day 2012-09-18 8.75 17.75
-- wrote unit test to reproduce problem report 2012-0098
-- fixed problem report 2012-0098
off 0.75
-- added HTML output option to object dumper
-- discussed implications of Java 1.7 rollout

day 2012-09-17 8 18
-- weekly team meeting
off 0.5
-- monthly quality task force 
off 0.25
-- code review: server side includes
\end{inputfile}

A log comment line starts in double dash characters \verb=--= and one or more space characters. All text following these characters until the end of the line (or until a \verb-#- source commend marker) with trailing spaces removed constitute the recorded log comment.

Now what can you do with such a file? Assume the above work log file's name is \verb:work-log.fly:, then option \verb:-t: tells \timeflies\ to calculate your work times.

\begin{inputfile}
> timeflies.py -t work-log.fly 
Time at work overview (all):
2012-09-17, Mon:  9.25 worked, --.-- leave, --.-- ill
2012-09-18, Tue:  8.25 worked, --.-- leave, --.-- ill
2012-09-19, Wed:  8.25 worked, --.-- leave, --.-- ill
   week 2012-38: 25.75 worked, --.-- leave, --.-- ill
  month 2012-09: 25.75 worked, --.-- leave, --.-- ill
          total: 25.75 worked, --.-- leave, --.-- ill
\end{inputfile}

To include the log comments in this output, use the \verb=-C= option:

\begin{inputfile}
> timeflies.py -t -C work-log.fly 
Time at work overview (all):
2012-09-17, Mon:  9.25 worked, --.-- leave, --.-- ill
             -- weekly team meeting
             -- monthly quality task force
             -- code review: server side includes
2012-09-18, Tue:  8.25 worked, --.-- leave, --.-- ill
             -- wrote unit test to reproduce problem report 2012-0098
             -- fixed problem report 2012-0098
             -- added HTML output option to object dumper
             -- discussed implications of Java 1.7 rollout
2012-09-19, Wed:  8.25 worked, --.-- leave, --.-- ill
             -- updated regression tests
             -- fixed build scripts
   week 2012-38: 25.75 worked, --.-- leave, --.-- ill
  month 2012-09: 25.75 worked, --.-- leave, --.-- ill
          total: 25.75 worked, --.-- leave, --.-- ill
\end{inputfile}

\subsection{Logging Activities}

Log comments are a good way to keep track of things you don't want to forget and have accessible and also aligned with your work time line.

What, if you need to make a connection between portions of your working time and work packages you have been working on? First, you define your work packages with \verb:work-package: instructions. Then you replace the relevant log comment lines by \emph{activity} lines in the day blocks. An activity line starts with a single dash character \verb:-: followed by one or more spaces. This is followed by a work package id that has been defined before and a duration. This is optionally followed by a comma and some activity comment.

\begin{inputfile}
work-package regression-tests
work-package meetings
work-package quality-task-force
work-package problem-reports
work-package development
work-package other

day 2012-09-19 8.5 17.25
- regression-tests 2, updated
off 0.5
- other 2, fixed build scripts

day 2012-09-18 8.75 17.75
- problem-reports 2, wrote unit test to reproduce problem report 2012-0098
- problem-reports 2.5, fixed problem report 2012-0098
off 0.75
- development 3, added HTML output option to object dumper
- other 1, discussed implications of Java 1.7 rollout

day 2012-09-17 8 18
- meetings 1.5, weekly team meeting
off 0.5
- quality-task-force 2 
off 0.25
- other 1, code review: server side includes
\end{inputfile}

Option \verb:-w: tells \timeflies\ to calculate the times you have been working on the different work packages:

\begin{inputfile}
> timeflies.py -w work-log.fly 
Work package summary (all):
 17.00 : ALL
      2.00 : regression-tests
      1.50 : meetings
      2.00 : quality-task-force
      4.50 : problem-reports
      3.00 : development
      4.00 : other
\end{inputfile}

To also show the activities contributing to the different work packages, use option \verb:-a::

\begin{inputfile}
> timeflies.py -w -a work-log.fly 
Work package summary (all):
 17.00 : ALL
      2.00 : regression-tests
             - 2012-09-19 2.0, updated
      1.50 : meetings
             - 2012-09-17 1.5, weekly team meeting
      2.00 : quality-task-force
             - 2012-09-17 2.0
      4.50 : problem-reports
             - 2012-09-18 2.0, wrote unit test to reproduce problem report 2012-0098
             - 2012-09-18 2.5, fixed problem report 2012-0098
      3.00 : development
             - 2012-09-18 3.0, added HTML output option to object dumper
      4.00 : other
             - 2012-09-19 2.0, fixed build scripts
             - 2012-09-18 1.0, discussed implications of Java 1.7 rollout
             - 2012-09-17 1.0, code review: server side includes
\end{inputfile}


%\begin{example}\begin{inputfile}
%day 2012-09-18 8.5 17.25
%-- wrote unit test to reproduce problem report 2012-0098
%-- fixed problem report 2012-0098
%off 0.75
%-- added HTML output option to object dumper
%off 0.25
%-- discussed implications of Java 1.7 rollout
%\end{inputfile}
%\caption{Day with log comments and off instructions}
%\end{example}

\subsection{Work Package Breakdown}

\subsection{Monitor Project Progress}

\section{Reference}


\end{document}

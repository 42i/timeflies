\documentclass[11pt]{article}
\usepackage{xunicode}
\usepackage{fontspec}
\usepackage{xltxtra}
\usepackage{float}
\usepackage{fancyvrb}
\defaultfontfeatures{Scale=MatchLowercase}
\setmainfont[Mapping=tex-text,Ligatures=TeX,Numbers=OldStyle]{Linux Libertine O}
\setsansfont[Mapping=tex−text,Numbers=OldStyle] {Linux Biolinum O}
\setmonofont[Mapping=tex-text,Scale=0.8] {DejaVu Sans Mono}

\newcommand{\timeflies}{\emph{TimeFlies}}

%\setromanfont{Linux Libertine O}

\begin{document}

\DefineVerbatimEnvironment{inputfile}{Verbatim}
   {%frame=single,
    %framesep=1mm,
    numbers=left,
    baselinestretch=0.8,
    labelposition=topline}

%\newfloat{example}{thp}{lof}[section]
%\floatname{example}{Example}

\title{\timeflies\ -- A Tool for Time Keeping}
\author{J\"org Bullmann}
\maketitle
\tableofcontents
\newpage

\section{What is \timeflies\ About?}

Do you need to account for the time you spend at work? What project or work package have you been working on? Do you want to keep track of the hours you work? How much leave have you got left for this year?

Do you keep a daily work log containing things you did, problems you solved, some kind of to do list?

Do you want to make an estimate of effort for a project or work package? Would you like to break down those things into smaller items and possibly break down those again too?

\timeflies\ can help you with this.

\section{Tutorial by Example}

In this section we will look at a number of use cases. All \timeflies\ data is kept in plain text files. So all your data is always easily accessible to you and the format itself is quite human-readable. Moreover it can easily be version controlled.

\subsection{Recording Time}

To record your work time keep a work log file with \verb-day- lines specifying the dates and in and out times telling when you arrived at work and when you left. The times can be given in decimals or in hours and minutes:

\input{fly-1.in}

Do you need to account for breaks you are taking? Use \verb-off- instructions to state periods of time in a day of work during which you were not actually working:

\input{fly-2.in}

This last file is equivalent to the following:

\input{fly-3.in}

This last example illustrates the notion of a \verb-day-\emph{-block}: a \verb-day-\emph{-block} extends
from one \verb-day--keyword to the next and everything inside this day block is part of that day.

The days in the file do not need to be listed chronologically. You could e.g.\ list the days in reverse order so that the present is always at the top of the file.

If you want to mask out part of your log temporarily you can use the \verb-#- source comment marker. \timeflies\ ignores the \verb-#- and everything following it until the end of line. It works just the same as e.g.\ a Python comment.

\input{fly-4.in}

\subsection{Keeping Notes in the Log}

Do you want to keep notes about your work in the same place as you keep the time information? Use \emph{log comment} lines like in this file:

\input{fly-5.in}

A log comment line starts with a semicolon and one or more space characters. All text following these characters until the end of the line (or until a \verb-#- source comment marker) with trailing spaces removed constitute the recorded log comment.

Now what can you do with such a file? Assume the above work log file's name is \verb:work-log.fly:, then option \verb:-t: tells \timeflies\ to calculate your work times.

\input{fly-5a.out}

To include the log comments in this output, use the \verb=-C= option:

\input{fly-5b.out}

\subsection{Time Summaries}

Here's a longer example where you can see the use of weekly and montly summaries. Also a few days of annual leave and sickness are inserted using the \verb:leave-days: and \verb:sick: instructions.

\input{fly-8.in}

Use option \verb:-f week: to get an overview of weekly work time balances.

\input{fly-8a.out}

Or have both weekly and monthly balances shown.

\input{fly-8b.out}

Maybe you only want to look at one month with daily details? Note the comments that have been associated with the leave and the sick days in the input file show up in the respective daily output lines.

\input{fly-8c.out}

You only need weekly totals in that one month?

\input{fly-8d.out}

\subsection{Logging Activities}

Log comments are a good way to keep track of things you don't want to forget and have accessible and also aligned with your work time line. Log comments have no work effort assigned to them, though. So you cannot use them in any way for calculations of effort spent.

You use \emph{work packages} and \emph{activities} to connect the time you work with the work packages you work on: first, you define your work packages, then you use \emph{activity} lines in the day blocks instead of log comment lines.

A work package definition is a line starting with the keyword \verb:work-package: (or its abbreviation \verb:wp:) followed by a work package name.

An activity line starts with a single dash character \verb:-: followed by one or more spaces. This is followed by a work package id and a duration. This is optionally followed by a semicolon and some activity comment.

See below the converted example work log file.

\input{fly-6.in}

Option \verb:-w: tells \timeflies\ to calculate the times you have been working on the different work packages:

\input{fly-6a.out}

To also show the activities contributing to the different work packages, use option \verb:-a::

\input{fly-6b.out}

To check whether you have allocated all your working time to work packages, use option \verb:-c::

\input{fly-6c.out}

This shows that on two days the time at work and the time worked on work packages are differing.

\subsection{Work Package Breakdown}

In the previous section, work packages have been defined as simple, atomic, named items. A work package can be subdivided and refined hierarchically. See the following example.

\input{fly-7-wp.in}

The items in this work package hierarchy can be referred to in activity lines as dot-delimited work package path names.

Following, a piece of work log for the above project.

\input{fly-7.in}

Assume file \verb:prj-mighty-digester.fly: contains the work package defintions and the work log itself is kept in \verb:work-log.fly:. The work package summary can be calculated with option \verb:-w: (which was also used in the previous example).

\input{fly-7a.out}

And here the same with activities shown.

\input{fly-7b.out}

%\subsection{Monitor Project Progress}

\section{Reference}

\subsection{Command Line Options}

Time filters

Summary filters

User filters

Show work packages

calculate work packages

check days

tally days

indentation

\subsection{File Syntax}

\subsubsection{Source Comments and Persistent Comments}

\timeflies\ knows two kinds of comments: \emph{source comments} and \emph{persistent comments}.

\begin{description}
\item[Source comments] are marked by a hash sign (\verb:#:) and extend from it to the end of the line. \timeflies\ ignores these comments and treats them as if they did not exist.

\item[Persistent comments] are marked by a semicolon (\verb:;:) and extend to the end of the line (or a possibly following source comment in that line). A persistent comment is processed and will show up in the generated output. E.g.\ comments on leave or sick days will show up in the time at work output summaries. Comments on activities or work packages can show up in the work package outputs. 
\end{description}

\subsubsection{Time and Date Formats}

Times and time durations can generally be given in \emph{[h]h:mm} format or in decimals. Examples would be \verb-8:30- or \verb-8.5-. Times use the 24-h-system. So 5:15 p.m.\ would therefore have to be written as either \verb-17:15- or \verb-17.25-.

Dates must generally be written in \emph{yyyy-mm-dd} format.

\subsubsection{Work Packages}

Syntax: \verb:work-package <wpid> [; <comment>]:.
The keyword \verb:work-package: can be abbreviated as \verb:wp:.

The fully qualified work package id \verb:wpid: is a dot-delimited sequence of simple work package ids (tokens consisting of alphanumeric characters). It resembles a path from the root of the work package hierarchy. In an compound id \emph{a.b} the id \emph{b} appearing directly to the right of \emph{a} means that work package \emph{b} is an immediate sub work package of \emph{a}. Work package hierarchies can therefore be given as sequences of work package definitions.

Alternatively (and more concise) a work package hierarchy can be given as hierarchically indented text (similar to Python indentation rules). In this case full work package ids are not necessary simple ones suffice.

Illustrating this, the follwing example:

\begin{inputfile}
wp pro; project of some sort
wp pro.aaa; part aaa
wp pro.bbb; part bbb
wp pro.bbb.xxx; detail xxx
wp pro.bbb.yyy; detail yyy
wp pro.ccc; part ccc
\end{inputfile}

is equivalent to:

\begin{inputfile}
wp pro; project of some sort
    aaa; part aaa
    bbb; part bbb
        xxx; detail xxx
        yyy; detail yyy
    ccc; part ccc
\end{inputfile}

This concise form of work package definition can also be applied partially. So the following is another form, equivalent to the above two:

\begin{inputfile}
wp pro; project of some sort
    aaa; part aaa
    bbb; part bbb
    ccc; part ccc
wp pro.bbb
    xxx; detail xxx
    yyy; detail yyy
\end{inputfile}

\subsubsection{Day blocks}

Syntax: \verb:day <date> [<in> <out>]:

A day block starts with a the keyword \verb:day: followed by the day's date and optionally the arrival and leaving times at work. Either both times must be given or none at all. If they are not given, no time at work is assumed for that day. This is used e.g. for leave days or sick days. A day block ends at the next \verb:day: keyword, i.e.\ at the beginning of a new day block.

\subsubsection{Activities}

Syntax: \verb:- <wpid> <time> [; <comment>]:

An \emph{activity} is a period of time spent working on a work package. An activity must appear in a day block. The work package id \verb:<wpid>: must be a valid fully qualified work package id, i.e. a matching work package must have been defined before.

\subsubsection{Time off Work}

Syntax: \verb:off <time>:

Declare times off work in a day block, yet inside the time span of that day.
 
\subsubsection{Sickness}

Syntax: \verb:sick <time> [; <comment>]:

Declare sick time in a day block, yet outside the time span of that day.

\subsubsection{Leave}

Syntax: \verb:leave <time> [; <comment>]:

Declare leave in a day block, yet outside the time span of that day. Use this for leave periods in order of hours.

\subsubsection{Block Leave (Several Days)}

Syntax: \verb:leave-days <first> <last> [; <comment>]:

A block of leave days is used to specify a period of more than one leave days. The leave block is defined by the first and last day of leave taken.

\end{document}

